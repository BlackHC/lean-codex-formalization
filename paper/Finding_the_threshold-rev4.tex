\documentclass[11pt,reqno]{amsart}

\usepackage[margin=1in]{geometry}
\usepackage[tt=false]{libertine}

\usepackage{amsmath,amsfonts,amssymb,amsthm}

\usepackage{mathabx}

\usepackage{mathrsfs}

\usepackage{graphicx}
\usepackage{enumerate}
\usepackage{bm}

\usepackage{url}
\usepackage{bbm}

\usepackage{verbatim}
\usepackage{hyperref,color}
\usepackage[capitalize,nameinlink]{cleveref}
\usepackage[dvipsnames]{xcolor}
\hypersetup{
	colorlinks=true,
	pdfpagemode=UseNone,
    citecolor=OliveGreen,
    linkcolor=NavyBlue,
    urlcolor=black,
	pdfstartview=FitW
}
\usepackage{appendix}
\crefname{appsec}{Appendix}{Appendices}
\usepackage{tikz}

\theoremstyle{plain}
\newtheorem{theorem}{Theorem}
\newtheorem{proposition}[theorem]{Proposition}
\newtheorem{lemma}[theorem]{Lemma}
\newtheorem{corollary}[theorem]{Corollary}
\newtheorem{conjecture}[theorem]{Conjecture}
\newtheorem*{conjecture*}{Conjecture}
\newtheorem{problem}[theorem]{Problem}
\newtheorem{claim}[theorem]{Claim}
\newtheorem{fact}[theorem]{Fact}

\theoremstyle{definition}
\newtheorem{definition}[theorem]{Definition}
\newtheorem{example}[theorem]{Example}
\newtheorem{observation}[theorem]{Observation}
\newtheorem{assumption}[theorem]{Assumption}
\newtheorem*{assumption*}{Assumption}

\theoremstyle{remark}
\newtheorem{remark}[theorem]{Remark}

\crefname{lemma}{Lemma}{Lemmas}
\crefname{theorem}{Theorem}{Theorems}
\crefname{definition}{Definition}{Definitions}
\crefname{fact}{Fact}{Facts}
\crefname{claim}{Claim}{Claims}
\crefname{proposition}{Proposition}{Propositions}

\newcommand{\dif}{\,\mathrm{d}}
\newcommand{\E}{\mathbb{E}}
\newcommand{\Var}{\mathrm{Var}}
\newcommand{\Cov}{\mathrm{Cov}}
\newcommand{\MI}{\mathrm{I}}
\newcommand{\Ent}{\mathrm{Ent}}
\newcommand{\one}{\mathbbm{1}}
\newcommand{\allone}{\mathbf{1}}
\newcommand{\tr}{\mathrm{tr}}
\newcommand{\wt}{\mathrm{weight}}
\DeclareMathOperator*{\argmax}{arg\,max}
\newcommand{\norm}[1]{\left\lVert #1 \right\rVert}
\newcommand{\TV}[2]{\left\|#1 - #2\right\|_{\mathrm{TV}}}
\newcommand{\ceil}[1]{\left\lceil #1 \right\rceil}
\newcommand{\floor}[1]{\left\lfloor #1 \right\rfloor}
\newcommand{\ip}[2]{\left\langle #1 , #2 \right\rangle}
\newcommand{\grad}{\nabla}
\newcommand{\hessian}{\nabla^2}
\newcommand{\trans}{\intercal}
\newcommand{\diag}{\mathrm{diag}}
\newcommand{\poly}{\mathrm{poly}}
\newcommand{\dist}{\mathrm{dist}}
\newcommand{\sgn}{\mathrm{sgn}}
\newcommand{\fpras}{\mathsf{FPRAS}}
\newcommand{\fpaus}{\mathsf{FPAUS}}
\newcommand{\fptas}{\mathsf{FPTAS}}

\newcommand{\eps}{\varepsilon}
\renewcommand{\epsilon}{\varepsilon}

\newcommand{\N}{\mathbb{N}}
\newcommand{\Q}{\mathbb{Q}}
\newcommand{\R}{\mathbb{R}}
\newcommand{\C}{\mathbb{C}}

\newcommand{\CC}{\mathcal{C}}
\newcommand{\XX}{\mathcal{X}}

\newcommand{\ccomp}{\mathsf{c}}


\newcommand{\QQ}{\mathbb{Q}}
\newcommand{\PP}{\mathbb{P}}
\newcommand{\EE}{\mathbb{E}}

\DeclareMathOperator{\Aut}{\mathsf{Aut}}

\newcommand{\beq}{\begin{equation}}
\newcommand{\eeq}{\end{equation}}

\newcommand{\ns}[1]{\textsf{\color{orange}[NS: #1]}}
\newcommand{\iz}[1]{\textsf{\color{blue}[IZ: #1]}}
\newcommand{\jnw}[1]{\textsf{\color{brown}[JNW: #1]}}

\raggedbottom


\newcommand{\bP}{\mathbf{P}}
\newcommand{\bQ}{\mathbf{Q}}
\newcommand{\bG}{\mathbf{G}}
\newcommand{\bA}{\mathbf{A}}
\newcommand{\bX}{\mathbf{X}}
\newcommand{\bY}{\mathbf{Y}}
\newcommand{\bV}{\mathbf{V}}
\newcommand{\bB}{\mathbf{B}}
\newcommand{\bI}{\mathbf{I}}


\linespread{1.1}

\begin{document}
	
\title{On the Second Kahn--Kalai Conjecture}

\author[E.\ Mossel, J.\ Niles-Weed, N.\ Sun, I.\ Zadik]{Elchanan Mossel$^{\star\circ}$, Jonathan Niles-Weed$^\dagger$, Nike Sun$^\star$, and Ilias Zadik$^\star$}
\thanks{$^\star$Department of Mathematics, MIT;
$^\circ$MIT Institute for Data, Systems, and Society;
$^\dagger$Center for Data Science \& Courant Institute of Mathematical Sciences, NYU. Email: \texttt{\{elmos,nsun,izadik\}@mit.edu}; \texttt{jnw@cims.nyu.edu}}

\date{ \vspace{0.2cm}
\today}


\newcommand{\pcrit}{p_\mathsf{c}}
\newcommand{\qfrac}{q_\mathsf{f}}
\newcommand{\pE}{p_\mathsf{E}}
\newcommand{\pEnew}{\tilde{p}_\mathsf{E}}

\begin{abstract} 
For any given graph $H$, we are interested in $\pcrit(H)$, the minimal $p$ such that the Erd\H{o}s--R\'enyi graph $G(n,p)$ contains a copy of $H$ with probability at least $1/2$. Kahn and Kalai (2007) conjectured that $\pcrit(H)$ is given up to a logarithmic factor by a simpler ``subgraph expectation threshold'' $\pE(H)$, which is the minimal $p$ such that for every subgraph $H'\subseteq H$, the Erd\H{o}s--R\'enyi graph $G(n,p)$ contains \emph{in expectation} at least $1/2$ copies of $H'$. It is trivial that $\pE(H)  \le  \pcrit(H)$, and the so-called ``second Kahn--Kalai conjecture'' states that $\pcrit(H) \lesssim \pE(H) \log e(H)$ where $e(H)$ is the number of edges in $H$.

In this article we present a natural modification $\pEnew(H)$ of the Kahn--Kalai subgraph expectation threshold, which we show is sandwiched between $\pE(H)$ and $\pcrit(H)$. The new definition $\pEnew(H)$ is based on the simple observation that if $G(n,p)$ contains a copy of $H$ and $H$ contains \emph{many} copies of $H'$, then $G(n,p)$ must also contain \emph{many} copies of $H'$. We then show that $\pcrit(H) \lesssim \pEnew(H) \log e(H)$, thus proving a modification of the second Kahn--Kalai conjecture. The bound follows by a direct application of the set-theoretic``spread''  property,  which led to recent breakthroughs in the sunflower conjecture by Alweiss, Lovett, Wu and Zhang and the first fractional Kahn--Kalai conjecture by Frankston, Kahn, Narayanan and Park. 


\end{abstract}

\maketitle

\section{Introduction}

In this work, we are interested in the following fundamental question. 
Given a graph $H$, what is the \textit{smallest} value $p=\pcrit(H)$ for which the Erd\H{o}s--R\'enyi graph $G=G(n,p)$ contains an isomorphic copy of $H$ with probability at least $1/2$?\footnote{The graph $H$ is allowed to depend on $n$. Indeed, when $e(H)$ does not grow with $n$ 
the value of $\pcrit(H)$ is known \cite{MR125031, bela_fixed}, so the main contribution of the present paper is in the setting where $e(H)$ grows with $n$.} The value 
$\pcrit(H)$ is often referred to as the ``critical threshold'' for the appearance of $H$. 
A well-known conjecture from \cite{kahn2007thresholds} posits that $\pcrit(H)$ 
is given up to a logarithmic factor by a simpler \emph{``subgraph expectation threshold''} $\pE(H)$, which is the \textit{maximum first-moment threshold among all subgraphs of $H$}.\footnote{Throughout, a ``subgraph'' of a graph $H$ refers to the \textit{edge-induced} subgraph associated with a subset of the edges of $H$.} More precisely, for any (labelled) graphs $H$ and $H’$, let $M_{H’,H}$ denote the number of subgraphs of $H$ which are isomorphic copies of $H’$, and define
	\beq\label{e:pE}
	\pE(H)
	=\min\bigg\{ p : \E M_{H',G(n,p)} \ge \frac12
	\textup{ for all } H' \subseteq H\bigg\}\,.
	\eeq
It is a trivial consequence of Markov's inequality {(see \S\ref{ss:basicobs})} that
$\pE(H)$ lower bounds $\pcrit(H)$. Kahn and Kalai proposed that this easy lower bound may not be far from the truth:

\begin{conjecture*}[{\cite[Conjecture~2]{kahn2007thresholds}}]
It holds that $\pcrit(H) \le L \pE(H) \log e(H)$ for a universal constant $L$.
\end{conjecture*} 


The factor $\log e(H)$ is necessary in some important examples, such as when $H$ is a perfect matching or Hamiltonian cycle. In both cases $\pE(H) \asymp 1/n$ but $\pcrit(H)\asymp (\log n)/n$ (\cite{MR125031,erdHos1966existence, MR389666,MR0434878}); for more details see the discussion in \cite{kahn2007thresholds}. The above conjecture remains open, although related conjectures of \cite{kahn2007thresholds,MR2743011} were proved recently (\cite{fracKK_annals,park2022proof}), {as we review below.}

\subsection{Main result}

In this article we introduce a natural variant $\pEnew(H)$ of $\pE(H)$, and show that it captures $\pcrit(H)$ up to a logarithmic factor, thus proving a modification of \cite[Conjecture~2]{kahn2007thresholds}. The modification is based on the simple observation that if $G=G(n,p)$ contains a copy of $H$, then we must have $M_{H',G} \ge M_{H',H}$ for any subgraph $H'$ of $H$ --- in contrast with the weaker bound $M_{H',G}\ge1$, which is used to show $\pE(H) \le \pcrit(H)$. Thus, if we define the  \emph{``modified subgraph expectation threshold''} 
	\beq\label{e:pEnew}
	\pEnew(H)
	=\min\bigg\{ p : \E M_{H',G(n,p)} \ge \frac{M_{H',H}}{2}
	\textup{ for all } H' \subseteq H\bigg\}\,,
	\eeq
then it is easy to see that $\pE(H) \le \pEnew(H) \le \pcrit(H)$ {(see also \S\ref{ss:basicobs} below).} Our main result is that the
new lower bound $\pEnew(H)\le\pcrit(H)$ is tight up to a logarithmic factor:

\begin{theorem}\label{thm:main}
It holds that $\pcrit(H) \le L \pEnew(H)\log e(H)$ for a universal constant $L$.
\end{theorem} 

A straightforward but tedious calculation gives that $\pEnew(H)\asymp 1/n$ when $H$ is a Hamiltonian cycle. Therefore, as with
\cite[Conjecture~2]{kahn2007thresholds}, the factor $\log e(H)$ is indeed necessary for Theorem~\ref{thm:main} to hold.

\subsection{Comparison with previous work}

The works most closely related to ours arise from the study of  \cite[Conjecture~1]{kahn2007thresholds}. 
 This ``first Kahn--Kalai conjecture'' applies more broadly to the setting of \emph{all monotone properties}, but is weaker than the second Kahn--Kalai conjecture (\cite[Conjecture~2]{kahn2007thresholds}) in the setting of \emph{graph inclusion properties} (which are the focus of this article). The first Kahn--Kalai conjecture states that for any monotone property $\mathscr{F}\subseteq\{0,1\}^X$, we have
	\[
	\pcrit(\mathscr{F}) \le q(\mathscr{F}) \log \ell(\mathscr{F})
	\]
where $q(\mathscr{F})$ is the \emph{maximum first moment threshold among all covers of $\mathscr{F}$}, and $\ell(\mathscr{F})$ is the size of a largest minimal element of $\mathscr{F}$. Talagrand \cite{MR2743011} proposed a relaxation of the above, the so-called ``fractional Kahn--Kalai conjecture''
	\[
	\pcrit(\mathscr{F}) 
	\le \qfrac(\mathscr{F}) \log \ell(\mathscr{F})
	\]
where $q(\mathscr{F})$ is the \emph{maximum first moment threshold among all fractional covers of $\mathscr{F}$}. It is trivial that $q(\mathscr{F})\le \qfrac(\mathscr{F})\le \pcrit(\mathscr{F})$. Both conjectures were long-standing open problems, which were resolved only recently in two notable works \cite{fracKK_annals,park2022proof}.




 In comparison, the second Kahn--Kalai conjecture is an even stronger conjecture in the particular setting of graph inclusion properties; Kahn and Kalai referred to it as their ``starting point'' in formulating their first conjecture. If $\mathscr{F}_H$ is the property of containing a copy of some graph $H$, then the threshold $\pE(H)$ of \eqref{e:pE} is the \emph{maximum first moment threshold among all ``subgraph containment covers'' of $\mathscr{F}_H$}. Therefore $\pE(H) \le q(\mathscr{F})$, and in the graph inclusion setting the first Kahn--Kalai conjecture may be viewed as a relaxation of the second. To the best of our knowledge, beyond the results on the first conjecture, no further progress has been made on the second one; and it has been reiterated in various places \cite{filmus2014real,fracKK_annals}. In this work, we modify and prove the second Kahn--Kalai conjecture (Theorem \ref{thm:main}). It is an interesting question whether Theorem~\ref{thm:main} can be used to prove (or disprove) the second Kahn--Kalai conjecture.

Interestingly, \emph{for graph inclusion properties}, our result slightly improves on the fractional Kahn--Kalai conjecture (posed by \cite{MR2743011} and proved by \cite{fracKK_annals} for general monotone properties). Indeed, our modified subgraph expectation threshold $\pEnew(H)$ can be interpreted as the 
\emph{maximum first moment threshold among certain
``subgraph containment fractional covers'' of $\mathscr{F}_H$}. To make the correspondence, using the notation of \cite{fracKK_annals}, for any subgraph $H'$ of $H$ one can assign weight $g_{H'}(S)=1/M_{H',H}$ to any subgraph $S\subseteq K_n$ that is a copy of $H'$. This leads to a fractional cover of $\mathscr{F}_H$ whose first-moment threshold is the unique $p$ satisfying $\E M_{H',G(n,p)} = M_{H',H}/2$.  It follows that $\pEnew(H) \le \qfrac(\mathscr{F}_H)$. Thus, our result implies the fractional Kahn--Kalai bound for graph inclusion properties, in fact with an explicit choice of fractional covers. Whether our result implies the original first Kahn--Kalai conjecture remains an interesting open problem.

\section{Proofs}

\subsection{Basic notations and calculations}
\label{ss:basicobs}
 Recall that $M_{H',H}$ denotes the number of (labelled) subgraphs of $H$ which are copies of $H'$. We abbreviate $M_H\equiv M_{H,K_n}$ where $K_n$ is the complete graph on $n$ vertices. We also abbreviate $Z_H\equiv M_{H,G}$ where $G$ is the Erd\H{o}s--R\'enyi graph $G(n,p)$. Writing $\PP_p$ for the law of $G(n,p)$, recall that
	\[
	\pcrit(H)
	= \inf\bigg\{ p : \PP_p(Z_H\ge1) \ge \frac12\bigg\}\,.
	\]
If $p \ge \pcrit(H)$, then Markov's inequality gives
	\[
	\frac12 \le \PP_p(Z_H\ge1) \le \PP_p\Big(Z_{H'}\ge1
		\ \forall H' \subseteq H\Big)
	\le \min\bigg\{ \E Z_{H'} : H'\subseteq H\bigg\}
	\,,
	\]
which implies $\pE(H) \le \pcrit(H)$ for $\pE(H)$ as defined by \eqref{e:pE}. Our definition \eqref{e:pEnew} of $\pEnew(H)$ is based on the simple observation that in fact $p \ge \pcrit(H)$ together with Markov's inequality implies
	\[
	\frac12 \le \PP_p(Z_H\ge1) \le \PP_p\Big(Z_{H'}\ge M_{H',H}
		\ \forall H' \subseteq H\Big)
	\le \min\bigg\{ \frac{\E Z_{H'}}{M_{H',H}}
	 : H'\subseteq H\bigg\}
	\,,
	\]
therefore $\pEnew(H) \le \pcrit(H)$. 
It is clear that
$\pE(H) \le\pEnew(H)$; moreover, if $\E_p$ denotes expectation with respect to $\PP_p$, then
 $\E_p Z_{H'} = M_{H'} p^{e(H')}$, so we can rewrite
 \eqref{e:pE} as
 	\[
	\pE(H)
	= \max\bigg\{
	\bigg(\frac{1}{2M_{H'}}\bigg)^{1/e(H')}
	: H'\subseteq H\bigg\}\,,
	\]
and likewise we can rewrite \eqref{e:pEnew} as
	\beq\label{e:gen.first.mmt.threshold}
	\pEnew(H)
	= \max\bigg\{
	\bigg(\frac{M_{H',H}}{2M_{H'}}\bigg)^{1/e(H')}
	: H'\subseteq H\bigg\}\,.
	\eeq


\subsection{The spread lemma} 
The proof of Theorem \ref{thm:main} is an application of a powerful tool, which we refer to as the ``spread lemma.'' Various forms of this lemma have played a key role in establishing recent breakthrough results on the sunflower theorem \cite{sunflower_annals} (see also \cite{Rao_sunflower, Tao_sunflower}) and the proof of the {fractional Kahn--Kalai conjecture}  \cite{fracKK_annals}.

To state the lemma in our setting, let $\pi$ be an arbitrary distribution over subgraphs of $K_n$ (e.g., the copies of particular subgraph of $K_n$). For $R>1$, we say that $\pi$ is \textit{$R$-spread} if for all (without loss of generality, nonempty) subgraphs $J_0\subseteq K_n$,  if $\bm{H}\sim\pi$ then
\beq\label{spread_cond}
\pi(J_0 \subseteq \bm{H}) \leq R^{-e(J_0)}.
\eeq
 Then the spread lemma as stated in \cite[Theorem 1.6]{fracKK_annals} applied to graph inclusion properties implies the following result.

\begin{lemma}\label{spread_lemma}
Fix integers $k,M \geq 1$. Let $G_1,\ldots,G_{M}$ be subgraphs of $K_n$ with $e(G_i) \leq k, i \in [M]$. For some universal constant $C>0$, if the uniform distribution $\pi$ over $\{G_1,\ldots,G_M\}$ is $R$-spread and  $p>C\frac{ \log k}{R} $, then a sample from $G(n,p)$ contains one of the $G_i$'s with probability at least $1/2$.
\end{lemma}

\begin{proof}We choose a sufficiently large constant $C>K$ where $K$ is the universal constant from \cite[Theorem 1.6]{fracKK_annals}. 
From standard concentration results, when $C>K$ is large enough, a sample from $G(n,p)$ contains, with probability at least $2/3$,
 a uniformly random undirected graph on $n$ vertices and $K\frac{ \log k}{R}\binom{n}{2}$ edges. The result then follows directly from \cite[Theorem 1.6]{fracKK_annals} applied to $X=K_n$ and $\kappa=R$.
\end{proof}

\subsection{The proof}

We now show that Theorem~\ref{thm:main} follows easily from Lemma~\ref{spread_lemma}:

\begin{proof}[Proof of Theorem~\ref{thm:main}]
Let $\pi\equiv\pi_H$ denote the uniform distribution over all the copies of $H$ in the complete graph $K_n$, and let $\bm{H}$ denote a sample from $\pi$. For a {nonempty} $J \subseteq H$, let $\pi_J$ denote the uniform distribution over all the copies of $J$ in $K_n$, and let $\bm{J}$ denote a sample from $\pi_J$. Let $J_0,H_0$ be any fixed copies of $J,H$ respectively in $K_n$. Then we have
	\beq\label{double_count} 
	\pi_H(J_0\subseteq\bm{H})
	=\pi_J( \bm{J} \subseteq H_0 )
	= \frac{M_{J,H}}{M_J}\,.
	\eeq
(The first equality is by symmetry. The second holds because there are $M_J$ possibilities for $\bm{J}$, of which $M_{J,H}$ are contained in $H_0$.) By combining \eqref{double_count}  with the definition \eqref{e:gen.first.mmt.threshold} of $\pEnew(H)$, we find
	\[
	\pi_H(J_0\subseteq\bm{H})
	\stackrel{\eqref{double_count}}{=}
		\frac{M_{J,H}}{M_J}
	\stackrel{\eqref{e:gen.first.mmt.threshold}}{\le}
		2 \pEnew(H)^{e(J)}
	\le \bigg(
	\frac1{2 \pEnew(H)}
	\bigg)^{-e(J)}\,,
	\]
where the last inequality uses that $e(J) \ge1$. Since $J_0$ is arbitrary, we conclude that $\pi$ is $R$-spread with
$R= 1/(2\pEnew(H))$. An appeal to Lemma \ref{spread_lemma} with $k=e(G)$ concludes the proof.
\end{proof}

\section*{Acknowledgements}
We thank Youngtak Sohn for helpful feedback on an earlier draft of this work. I.Z. thanks Dan Mikulincer for helpful discussions. 
We also acknowledge the support of
Simons-NSF grant DMS-2031883 (E.M., N.S., and I.Z.),
the Vannevar Bush Faculty Fellowship ONR-N00014-20-1-2826 (E.M.\ and I.Z.), the Simons Investigator Award 622132 (E.M.),
the Sloan Research Fellowship (J.N.W.),
 and NSF CAREER grant DMS-1940092 (N.S.).

\vfill

{\raggedright
\bibliographystyle{alphaabbr}
\bibliography{pc}}
\end{document}